\documentclass[journal,12pt,twocolumn]{IEEEtran}

\usepackage{setspace}
\usepackage{gensymb}

\singlespacing


\usepackage[cmex10]{amsmath}

\usepackage{amsthm}

\usepackage{mathrsfs}
\usepackage{txfonts}
\usepackage{stfloats}
\usepackage{bm}
\usepackage{cite}
\usepackage{cases}
\usepackage{subfig}

\usepackage{longtable}
\usepackage{multirow}

\usepackage{enumitem}
\usepackage{mathtools}
\usepackage{steinmetz}
\usepackage{tikz}
\usepackage{circuitikz}
\usepackage{verbatim}
\usepackage{tfrupee}
\usepackage[breaklinks=true]{hyperref}
\usepackage{graphicx}
\usepackage{tkz-euclide}
\usepackage{float}

\usetikzlibrary{calc,math}
\usepackage{listings}
    \usepackage{color}                                            %%
    \usepackage{array}                                            %%
    \usepackage{longtable}                                        %%
    \usepackage{calc}                                             %%
    \usepackage{multirow}                                         %%
    \usepackage{hhline}                                           %%
    \usepackage{ifthen}                                           %%
    \usepackage{lscape}     
\usepackage{multicol}
\usepackage{chngcntr}

\DeclareMathOperator*{\Res}{Res}

\renewcommand\thesection{\arabic{section}}
\renewcommand\thesubsection{\thesection.\arabic{subsection}}
\renewcommand\thesubsubsection{\thesubsection.\arabic{subsubsection}}

\renewcommand\thesectiondis{\arabic{section}}
\renewcommand\thesubsectiondis{\thesectiondis.\arabic{subsection}}
\renewcommand\thesubsubsectiondis{\thesubsectiondis.\arabic{subsubsection}}


\hyphenation{op-tical net-works semi-conduc-tor}
\def\inputGnumericTable{}                                 %%

\lstset{
%language=C,
frame=single, 
breaklines=true,
columns=fullflexible
}
\begin{document}


\newtheorem{theorem}{Theorem}[section]
\newtheorem{problem}{Problem}
\newtheorem{proposition}{Proposition}[section]
\newtheorem{lemma}{Lemma}[section]
\newtheorem{corollary}[theorem]{Corollary}
\newtheorem{example}{Example}[section]
\newtheorem{definition}[problem]{Definition}

\newcommand{\BEQA}{\begin{eqnarray}}
\newcommand{\EEQA}{\end{eqnarray}}
\newcommand{\define}{\stackrel{\triangle}{=}}
\bibliographystyle{IEEEtran}
\providecommand{\mbf}{\mathbf}
\providecommand{\pr}[1]{\ensuremath{\Pr\left(#1\right)}}
\providecommand{\qfunc}[1]{\ensuremath{Q\left(#1\right)}}
\providecommand{\sbrak}[1]{\ensuremath{{}\left[#1\right]}}
\providecommand{\lsbrak}[1]{\ensuremath{{}\left[#1\right.}}
\providecommand{\rsbrak}[1]{\ensuremath{{}\left.#1\right]}}
\providecommand{\brak}[1]{\ensuremath{\left(#1\right)}}
\providecommand{\lbrak}[1]{\ensuremath{\left(#1\right.}}
\providecommand{\rbrak}[1]{\ensuremath{\left.#1\right)}}
\providecommand{\cbrak}[1]{\ensuremath{\left\{#1\right\}}}
\providecommand{\lcbrak}[1]{\ensuremath{\left\{#1\right.}}
\providecommand{\rcbrak}[1]{\ensuremath{\left.#1\right\}}}
\theoremstyle{remark}
\newtheorem{rem}{Remark}
\newcommand{\sgn}{\mathop{\mathrm{sgn}}}
\providecommand{\abs}[1]{\lvert#1\vert}
\providecommand{\res}[1]{\Res\displaylimits_{#1}} 
\providecommand{\norm}[1]{\lVert#1\rVert}
%\providecommand{\norm}[1]{\lVert#1\rVert}
\providecommand{\mtx}[1]{\mathbf{#1}}
\providecommand{\mean}[1]{E[ #1 ]}
\providecommand{\fourier}{\overset{\mathcal{F}}{ \rightleftharpoons}}
%\providecommand{\hilbert}{\overset{\mathcal{H}}{ \rightleftharpoons}}
\providecommand{\system}{\overset{\mathcal{H}}{ \longleftrightarrow}}
	%\newcommand{\solution}[2]{\textbf{Solution:}{#1}}
\newcommand{\solution}{\noindent \textbf{Solution: }}
\newcommand{\cosec}{\,\text{cosec}\,}
\providecommand{\dec}[2]{\ensuremath{\overset{#1}{\underset{#2}{\gtrless}}}}
\newcommand{\myvec}[1]{\ensuremath{\begin{pmatrix}#1\end{pmatrix}}}
\newcommand{\mydet}[1]{\ensuremath{\begin{vmatrix}#1\end{vmatrix}}}
\numberwithin{equation}{subsection}
\makeatletter
\@addtoreset{figure}{problem}
\makeatother
\let\StandardTheFigure\thefigure
\let\vec\mathbf
\renewcommand{\thefigure}{\theproblem}
\def\putbox#1#2#3{\makebox[0in][l]{\makebox[#1][l]{}\raisebox{\baselineskip}[0in][0in]{\raisebox{#2}[0in][0in]{#3}}}}
     \def\rightbox#1{\makebox[0in][r]{#1}}
     \def\centbox#1{\makebox[0in]{#1}}
     \def\topbox#1{\raisebox{-\baselineskip}[0in][0in]{#1}}
     \def\midbox#1{\raisebox{-0.5\baselineskip}[0in][0in]{#1}}
\vspace{3cm}
\title{ASSIGNMENT-13}
\author{Unnati Gupta}
\maketitle
\newpage
\bigskip
\renewcommand{\thefigure}{\theenumi}
\renewcommand{\thetable}{\theenumi}
Download all python codes from 
\begin{lstlisting}
https://github.com/unnatigupta2320/Assignment_13
\end{lstlisting}
%
and latex-tikz codes from 
%
\begin{lstlisting}
https://github.com/unnatigupta2320/Assignment_13
\end{lstlisting}
%
\section{Question No-6.20}
An unbiased dice is thrown twice. Let the event A be "odd number on the first throw" and B be "Odd number on second throw". Check the independence of event A and B. 
\section{Solution}
\begin{lemma}
Two events are independent if knowing one event occurred doesn't change the probability of the other event.
\\
$\therefore$ A and B are said to be independent if:
\begin{align}
 P(A \cap B)=P(A)P(B)   
\end{align}
\label{lemma1}
\end{lemma} 
\begin{enumerate}
\item According to given data we have,
\begin{table}[ht!]
\begin{tabular}{|l|l|}
\hline
\textbf{Events} & \textbf{Description}                         \\ \hline
A & Odd number on first throw
\\ \hline
B & Odd number on second throw 
\\ \hline
$A\cap B$ & Odd Numbers appears on both throw  \\ \hline
\end{tabular}
\end{table}
\item When a die is thrown twice the possible outcomes are:
\begin{multline}
  S=\{\brak{1,1},\brak{1,2},\brak{1,3},\brak{1,4},\brak{1,5},\brak{1,6},
\\\brak{2,1},\brak{2,2},\brak{2,3},\brak{2,4},\brak{2,5},\brak{2,6},
\\\brak{3,1},\brak{3,2},\brak{3,3},\brak{3,4},\brak{3,5},\brak{3,6},
\\\brak{4,1},\brak{4,2},\brak{4,3},\brak{4,4},\brak{4,5},\brak{4,6},
\\\brak{5,1},\brak{5,2},\brak{5,3},\brak{5,4},\brak{5,5},\brak{5,6},
  \\\brak{6,1},\brak{6,2},\brak{6,3},\brak{6,4},\brak{6,5},\brak{6,6}\}
\end{multline}

\item For the event \textbf{A} : Odd number on first throw
\begin{itemize}
\item The Sample space is:
\begin{multline}
    S_A=
    \\
    \{\brak{1,1},\brak{1,2},\brak{1,3},\brak{1,4},\brak{1,5},\brak{1,6},\\\brak{3,1},\brak{3,2},\brak{3,3},\brak{3,4},\brak{3,5},\brak{3,6},\\\brak{5,1},\brak{5,2},\brak{5,3},\brak{5,4},\brak{5,5},\brak{5,6}\}
    \end{multline}
    \item So, the probability of odd number on first throw is-
\begin{align}
\implies  Pr(A) &= \frac{18}{36} 
\\
\implies  Pr(A) &= \frac{1}{2} \label{eqA}
 \end{align}
 \end{itemize}
 \item For the event \textbf{B}: Odd number on second throw
    \begin{itemize}
    \item The Sample space is:
 \begin{multline}
    S_B= 
    \\
    \{\brak{1,1},\brak{2,1},\brak{3,1},\brak{4,1},\brak{5,1},\brak{6,1},\\\brak{1,3},\brak{2,3},\brak{3,3},\brak{4,3},\brak{5,3},\brak{6,3},\\\brak{1,5},\brak{2,5},\brak{3,5},\brak{4,5},\brak{5,5},\brak{6,5}\}
    \end{multline}
    \item So, the probability of odd number on second throw is-
\begin{align}
\implies  Pr(B) &= \frac{18}{36} 
\\
\implies  Pr(B) &= \frac{1}{2} \label{eqB}
 \end{align}
 \end{itemize}
\item For,$A\cap B$ : Odd Numbers appears on both throw
\begin{itemize}
\item The Sample space is:
\begin{multline}
    S=\{\brak{1,1},\brak{1,3},\brak{1,5},\\\brak{3,1},\brak{3,3},\brak{3,5},\\\brak{5,1},\brak{5,3},\brak{5,5}\}
    \end{multline}
    \item So, the probability that odd numbers appears on both throw is-
\begin{align}
\implies  Pr(A\cap B) &= \frac{9}{36} 
\\
\implies  Pr(A\cap B) &= \frac{1}{4} \label{eqC}
 \end{align}
 \end{itemize}
\item Now to check whether the events are \textbf{independent}, we use Lemma \eqref{lemma1}.
\begin{align}
 \implies   Pr(A\cap B) = Pr(A)Pr(B) 
    \end{align}
    \item Putting values from \eqref{eqA} and \eqref{eqB} we get,
    \begin{align}
  \implies  Pr(A\cap B) & = \frac{1}{2}\times\frac{1}{2}
    \\
\implies   Pr(A\cap B)  &= \frac{1}{4}
\end{align}
This is equal to value in equation \eqref{eqC}.
\\
Hence, the events are \textbf{independent}.
\end{enumerate}
\end{document}
